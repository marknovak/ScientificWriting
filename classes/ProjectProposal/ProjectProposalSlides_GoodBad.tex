
\documentclass[10pt]{beamer}

\usetheme{default}
\setbeamertemplate{navigation symbols}{}

\title{Project Proposal Examples}
\subtitle{Strong vs.\ Weak Proposals}
\author{IB 514: Graduate Scientific Writing\\Mark Novak}
\date{}

\begin{document}
	
	\begin{frame}
		\titlepage
		
		\textbf{3-slide limit:} \\
		Project Overview, Structure \& Scope, Journal Strategy.
	\end{frame}
	
	% -------------------------------------------------
	\section*{Slide 1: Project Overview}
	
	\begin{frame}{Slide 1: Project Overview — Strong Example}
		
		\textbf{Working Title}\\
		Growth-induced anti-switching destabilizes coastal predator-prey interactions
		
		\vspace{0.5em}
		
		\textbf{Likely co-authors}\\
		Me, Lila Yang, Malcolm Peralta, Malayah York

		\vspace{0.5em}
		
		\textbf{Project Description}\\
		We characterize prey switching by generalist predators and how it destabilizes rocky intertidal 
		population dynamics using long-term observational data and dynamic models.
		
		\vspace{0.5em}
		
		\textbf{Main Message}\\
		Ontogenetic increases in predator body size affect population-level anti-switching and destabilize predator-prey dynamics.

		\vspace{0.5em}
		
		\textbf{Manuscript Type}\\
		Primary research article
		
		\vspace{0.5em}
		
		\textbf{Intended Audience}\\
		Community ecologists, both empiricists and theoreticians
		
		\vspace{0.75em}
		
		\textit{Why this works:} Clear contribution, specific scope, identifiable audience.
		
	\end{frame}
	
	\begin{frame}{Slide 1: Project Overview — Weak Example}
		
		\textbf{Working Title}\\
		Food webs and predator behavior
		
		\vspace{0.5em}
		
		\textbf{Project Description}\\
		This paper looks at predator-prey interactions and how they affect ecosystems. I will explore
		some data and models.
		
		\vspace{0.5em}
		
		\textbf{Likely co-authors}\\
		Me, my advisor

		\vspace{0.5em}

		\textbf{Manuscript Type}\\
		Chapter (hopefully publishable)
		
		\vspace{0.5em}
		
		\textbf{Intended Audience}\\
		Ecologists
		
		\vspace{0.75em}
		
		\textit{Why this is weak:} Vague question, undefined contribution, generic audience.
		
	\end{frame}
	
	% -------------------------------------------------
	\section*{Slide 2: Structure, Scope, and Feasibility}
	
	\begin{frame}{Slide 2: Structure \& Scope — Strong Example}
		
		\textbf{Proposed Structure}
		\begin{itemize}
			\item Intro: prey switching theory predictions and assumptions
			\item Methods: long-term intertidal surveys, observational approach, dynamical model
			\item Results: population-level anti-switching caused by the growth of individuals
			\item Discussion: implications for food web stability
			\item Supplementary: mathematical details, lab experiment
		\end{itemize}
		
		\vspace{0.5em}
		
		\textbf{Data and Analyses}
		\begin{itemize}
			\item 3-year monthly species abundance and predator diet surveys
			\item Existing model \& code adapted from prior published work
		\end{itemize}
		
		\vspace{0.5em}
		
		\textbf{Scope}
		\begin{itemize}
			\item Will include lab experiments (as supplementary materials)
			\item Won't include the field experiment or literature synthesis 
		\end{itemize}
		
		\vspace{0.75em}
		
		\textit{Why this works:} Concrete structure, feasible analyses, clear limits.
		
	\end{frame}
	
	\begin{frame}{Slide 2: Structure \& Scope — Weak Example}
		
		\textbf{Proposed Structure}
		\begin{itemize}
			\item Introduction
			\item Results
			\item Discussion
		\end{itemize}
		
		\vspace{0.5em}
		
		\textbf{Data and Analyses}
		\begin{itemize}
			\item Various datasets
			\item Some models
		\end{itemize}
		
		\vspace{0.5em}
		
		\textbf{Scope}
		\begin{itemize}
			\item Depends on how things go
		\end{itemize}
		
		\vspace{0.75em}
		
		\textit{Why this is weak:} Non-informative structure, undefined data, no feasibility signal.
		
	\end{frame}
	
	% -------------------------------------------------
	\section*{Slide 3: Journal Strategy and Timeline}
	
	\begin{frame}{Slide 3: Journal Strategy — Strong Example}
		
		\textbf{Candidate Journals}
		\begin{enumerate}
			\item Aspire: \textit{Ecology Letters} — widely read by empiricists \& theoreticians, general ecology
			\item Solid: \textit{Journal of Animal Ecology} — strong fit for predator behavior and dynamics
			\item Backup: \textit{Marine Ecology} — solid fit for marine audience
		\end{enumerate}
		
		\vspace{0.5em}
		
		\textbf{Format and Constraints for Aspiration}
		\begin{itemize}
			\item $\sim$6,000 words main text (incl. captions), 200 word abstract, IMRaD required, 5 figures, allows supplemental materials
		\end{itemize}
		
		\vspace{0.5em}
		
	
		\small{
		\textbf{Timeline}
		\begin{columns}
			\column{0.5\textwidth}
				\begin{itemize}
					\item Methods draft: Wk 3
					\item Results draft: Wk 4
					\item Complete analyses: Wk 5
					\item Intro draft: Wk 6
				\end{itemize}
			\column{0.5\textwidth}
			\begin{itemize}
					\item Discussion draft: Wk 7
					\item Full draft to co-authors: Wk 8
					\item Submission-ready draft: Wk 10
					\item Submission: Wk 11
				\end{itemize}
		\end{columns}
		}
		\vspace{0.75em}
		
		\textit{Why this works:} Strategically-ranked journals, knowledge of format and constraints, realistic timeline.
		
	\end{frame}
	
	\begin{frame}{Slide 3: Journal Strategy — Weak Example}
		
		\textbf{Candidate Journals}
		\begin{itemize}
			\item \textit{Science}
			\item \textit{Nature}
			\item \textit{PeerJ}
		\end{itemize}
		
		\vspace{0.5em}
		
		\textbf{Format Constraints}
		\begin{itemize}
			\item Not sure yet
		\end{itemize}
		
		\vspace{0.5em}
		
		\textbf{Timeline}
		\begin{itemize}
			\item Write paper
			\item Submit
		\end{itemize}
		
		\vspace{0.75em}
		
		\textit{Why this is weak:} Prestige-driven targeting, no evidence of planning or realism.
		
	\end{frame}
	
	% -------------------------------------------------
	\section*{Teaching Note}
	
	\begin{frame}{Reflection}
		
		\begin{itemize}
			\item What information is essential to the message vs.\ optional?
			\item Where do the strong examples signal feasibility?
			\item How do journal choices reflect audience rather than ambition?
		\end{itemize}
		

			\vspace{1em}

		Plan to revise your slides (or notes) after peer feedback.
		
	\end{frame}
	
\end{document}
