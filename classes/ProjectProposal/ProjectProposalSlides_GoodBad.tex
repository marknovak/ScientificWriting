
\documentclass{beamer}

\usetheme{default}
\setbeamertemplate{navigation symbols}{}

\title{Project Proposal Examples}
\subtitle{Good vs.\ Weak Proposals}
\author{IB 514: Graduate Scientific Writing\\Mark Novak}
\date{}

\begin{document}
	
	\begin{frame}
		\titlepage
		
		\textbf{3-slide limit:} \\
		Project Overview, Structure \& Scope, Journal Strategy.
	\end{frame}
	
	% -------------------------------------------------
	\section*{Slide 1: Project Overview}
	
	\begin{frame}{Slide 1: Project Overview — Strong Example}
		
		\textbf{Working Title}\\
		Context-dependent predator switching stabilizes coastal food webs
		
		\vspace{0.5em}
		
		\textbf{Project Description}\\
		We test whether prey switching by generalist predators stabilizes rocky intertidal food webs
		across environmental gradients using long-term observational data and dynamic models.
		
		\vspace{0.5em}
		
		\textbf{Manuscript Type}\\
		Research article
		
		\vspace{0.5em}
		
		\textbf{Intended Audience}\\
		Community ecologists and food web modelers
		
		\vspace{0.75em}
		
		\textit{Why this works:} Clear contribution, specific scope, identifiable audience.
		
	\end{frame}
	
	\begin{frame}{Slide 1: Project Overview — Weak Example}
		
		\textbf{Working Title}\\
		Food webs and predator behavior
		
		\vspace{0.5em}
		
		\textbf{Project Description}\\
		This paper looks at predator-prey interactions and how they affect ecosystems. I will explore
		some data and models.
		
		\vspace{0.5em}
		
		\textbf{Manuscript Type}\\
		Paper
		
		\vspace{0.5em}
		
		\textbf{Intended Audience}\\
		Ecologists
		
		\vspace{0.75em}
		
		\textit{Why this is weak:} Vague question, undefined contribution, generic audience.
		
	\end{frame}
	
	% -------------------------------------------------
	\section*{Slide 2: Structure, Scope, and Feasibility}
	
	\begin{frame}{Slide 2: Structure \& Scope — Strong Example}
		
		\textbf{Proposed Structure}
		\begin{itemize}
			\item Introduction: prey switching theory and predictions
			\item Methods: long-term intertidal surveys + dynamical model
			\item Results: switching strength across gradients
			\item Discussion: implications for food web stability
			\item Supplementary: mathematical details
		\end{itemize}
		
		\vspace{0.5em}
		
		\textbf{Data and Analyses}
		\begin{itemize}
			\item 12-year species abundance dataset
			\item Existing model \& code adapted from prior work
		\end{itemize}
		
		\vspace{0.5em}
		
		\textbf{Explicit Boundaries}
		\begin{itemize}
			\item No experimental manipulation
			\item No evolutionary dynamics
		\end{itemize}
		
		\vspace{0.75em}
		
		\textit{Why this works:} Concrete structure, feasible analyses, clear limits.
		
	\end{frame}
	
	\begin{frame}{Slide 2: Structure \& Scope — Weak Example}
		
		\textbf{Proposed Structure}
		\begin{itemize}
			\item Introduction
			\item Results
			\item Discussion
		\end{itemize}
		
		\vspace{0.5em}
		
		\textbf{Data and Analyses}
		\begin{itemize}
			\item Various datasets
			\item Some models
		\end{itemize}
		
		\vspace{0.5em}
		
		\textbf{Scope}
		\begin{itemize}
			\item Depends on how things go
		\end{itemize}
		
		\vspace{0.75em}
		
		\textit{Why this is weak:} Non-informative structure, undefined data, no feasibility signal.
		
	\end{frame}
	
	% -------------------------------------------------
	\section*{Slide 3: Journal Strategy and Timeline}
	
	\begin{frame}{Slide 3: Journal Strategy — Strong Example}
		
		\textbf{Candidate Journals}
		\begin{enumerate}
			\item \textit{Ecology Letters} — empiricists \& theoreticians, general ecology
			\item \textit{Journal of Animal Ecology} — strong fit for predator behavior and dynamics
			\item \textit{Marine Ecology} — solid fit for marine audience
		\end{enumerate}
		
		\vspace{0.5em}
		
		\textbf{Format Constraints}
		\begin{itemize}
			\item $\sim$6,000 words main text
			\item 4--5 figures, allows supplemental online materials
		\end{itemize}
		
		\vspace{0.5em}
		
		\textbf{Timeline}
		\begin{itemize}
			\item Complete analyses: Week 5
			\item Full draft: Week 7
			\item Submission-ready draft: Week 10
		\end{itemize}
		
		\vspace{0.75em}
		
		\textit{Why this works:} Strategic journal choice, realistic timeline, awareness of constraints.
		
	\end{frame}
	
	\begin{frame}{Slide 3: Journal Strategy — Weak Example}
		
		\textbf{Candidate Journals}
		\begin{itemize}
			\item \textit{Science}
			\item \textit{Nature}
			\item \textit{PeerJ}
		\end{itemize}
		
		\vspace{0.5em}
		
		\textbf{Format Constraints}
		\begin{itemize}
			\item Not sure yet
		\end{itemize}
		
		\vspace{0.5em}
		
		\textbf{Timeline}
		\begin{itemize}
			\item Write paper
			\item Submit
		\end{itemize}
		
		\vspace{0.75em}
		
		\textit{Why this is weak:} Prestige-driven targeting, no evidence of planning or realism.
		
	\end{frame}
	
	% -------------------------------------------------
	\section*{Teaching Note}
	
	\begin{frame}{Reflection}
		
		\begin{itemize}
			\item What information is essential vs.\ optional?
			\item Where does the strong example signal feasibility?
			\item How do journal choices reflect audience rather than ambition?
		\end{itemize}
		
		I encourage you to revise their slides immediately after peer feedback.
		
	\end{frame}
	
\end{document}
