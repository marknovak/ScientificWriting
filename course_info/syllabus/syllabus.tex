\documentclass[10pt]{article}

\usepackage{titlesec}
\titlespacing\section{0pt}{5pt}{2pt}
\titlespacing\subsection{0pt}{5pt}{2pt}
%\setlength{\parindent}{0pt} % remove automatic indentation


\usepackage[text={6.5in,8.5in},centering]{geometry}
\geometry{verbose,a4paper,tmargin=2.4cm,bmargin=2.4cm,lmargin=2.4cm,rmargin=2.4cm}

\usepackage{soul} % for \st{} strike-out
\usepackage{hyperref} % for activation hyperlinks
\hypersetup{colorlinks = true, allcolors = blue  }

\usepackage{enumitem}



%%%%%%%%%%%%%%%%%%%%%%%%%%%%%%%

\title{IB 514 SCIENTIFIC WRITING}

\author{}
\date{}

%%%%%%%%%%%%%%%%%%%%%%%%%%%%%%%

\begin{document}
\maketitle
\vspace{-50pt}


\section*{Course Description}
	TBD

\section*{Prerequisites}
\noindent
	Officially, only graduate standing (or instructor approval), but see end of 
	\emph{Course Description} for what you should have coming in.

\section*{Learning Outcomes}
\noindent
After successful completion of this course, you should be able to:
\begin{enumerate}
	\itemsep0em
	\item 
	\item 
	\item 
	\item 
\end{enumerate}

\section*{Instructor}
	\noindent
	Mark Novak\\
	Office: Cordley 5323\\
	Email: \href{mailto:mark.novak@oregonstate.edu}{mark.novak@oregonstate.edu}\\


\section*{Office Hours}
\noindent
	I am happy to meet by appointment, or please feel free to stop by my office 
	anytime.

\section*{Meeting Times, Location, Course Materials \& Schedule}
\noindent
See
\href{https://github.com/ScientificWriting}{https://github.com/ScientificWriting}
 

%\clearpage

\section*{Course Work}
There will be no exams, tests, or quizzes.
I will not grade your work because there won't be ``correct'' answers to the 
problems you will solve.
In fact, my goal is for you to not have almost no ``homework'' (besides readings 
and making progress on your manuscript writing).
However, the course \textit{will} require a significant allocation of time for 
thought and reflection outside of class.
There \textit{will} also be readings for in-class discussion to be read before class (see Schedule on GitHub repository).
Finally, you will need to complete all assigned in-class work, ideally by the end 
of the week in which the topic is covered but definitely before the end of Week 
10.

\section*{Evaluation of Student Performance}
Grading (A-F) will be based on your participation, your project presentations 
(see rubrics on the GitHub repository),
and the completion of the various in-class ``tasks'' assigned throughout the quarter.
Accommodations for missed classes (e.g., due to fieldwork) are okay in principle, as long as we can identify a path to your having fully participated in the course.
Please discuss your specific situation with me during Week 1.
Everyone who puts the effort into the class will get a grade that will make them happy (if you, as a graduate student, still care about grades).

\section*{Collaboration}
My hope is for you to make a tremendous amount of progress on your 
manuscript during this course.
Your work will therefore, almost by definition and necessity, be your own.
That said, I \textit{strongly} encourage everyone to work as a team in
whatever ways are possible!
That's because what this course is all about is to provide you with tools.
Tools have to be learned and practiced.
Moreover, there's nothing better for learning to understand something than 
(successfully) explaining how you think about it to someone else.
Therefore, you are \textit{strongly} encouraged to collaborate.
This will be emphasized throughout the quarter by getting you to work or 
discuss your work in pairs or teams.

\section*{Student Learning Experience Survey}
The online Student Learning Experience surveys will open to you the 
Wednesday of week 9 and close the Sunday before Finals Week. 
You will receive notification, instructions, and the link through your ONID email. 
You may also log into the survey via MyOregonState or directly at 
\url{https://beav.es/Student-Learning-Survey}. 

\noindent
\textit{
\textbf{Survey results are extremely important to how my department values 
this course.
They also help me improve the course and the learning experience of future 
students.}}

Responses are anonymous (unless you choose to ``sign'' your comments, 
agreeing to relinquish anonymity of written comments) and are not available to 
instructors until after grades have been posted. 
I've been told that the results of scaled (quantitative) questions and signed 
comments go to me and the Head 
of my department.
Anonymous (unsigned) comments go only to me.

%\clearpage


\section*{Policies}
\tiny
\subsection*{Academic Calendar}
All students are subject to the registration and refund deadlines as stated in the Academic Calendar: 
\url{https://registrar.oregonstate.edu/osu-academic-calendar}

\subsection*{Equity, Justice and Inclusion}
Oregon State University, and the Department of Integrative Biology, have a lot 
of work to do on improving the level of equity, justice and inclusion in our 
community. 
I will take responsibility for creating an welcoming and supportive climate for 
everyone in the course. 
If you have concerns or suggestions related to these issues as they pertain to 
the course, please do not hesitate to contact me and/or the Office for 
Institutional Diversity
(\href{https://diversity.oregonstate.edu/}{https://diversity.oregonstate.edu/}).

\subsection*{Student Conduct Expectations}
Choosing to join the Oregon State University community obligates each 
member to a code of responsible behavior, which is outlined in the Student 
Conduct Code
(\href{https://beav.es/codeofconduct/}{ https://beav.es/codeofconduct/}).
This Code is based on the assumption that all persons must treat one another 
with dignity and respect in order for scholarship to thrive.

\subsection*{Students with Disabilities}
Accommodations for students with disabilities are determined and approved by 
Disability Access Services (DAS).
If you, as a student, believe you are eligible for accommodations but have not 
obtained official approval, please contact DAS at 541-737-4098 or at
\href{http://ds.oregonstate.edu}{http://ds.oregonstate.edu}.
DAS notifies students and faculty members of approved academic 
accommodations and coordinates implementation of those accommodations.
While not required, students and faculty members are encouraged to discuss 
details of the implementation of individual accommodations.

\subsection*{Reach Out for Success}
University students encounter setbacks from time to time. 
If you encounter difficulties and need assistance, it’s important to reach out. 
Consider discussing the situation with an instructor or academic advisor. 
Learn about resources that assist with wellness and academic success at 
oregonstate.edu/ReachOut. 
If you are in immediate crisis, please contact the Crisis Text Line by texting 
OREGON to 741-741 or call the National Suicide Prevention Lifeline at 
1-800-273-TALK (8255).

\subsection*{Religious Holidays}
Oregon State University strives to respect all religious practices.
If you have religious holidays that are in conflict with any of the requirements of 
this class, please see me as soon as you can so that we can make alternative 
arrangements.




\end{document}
