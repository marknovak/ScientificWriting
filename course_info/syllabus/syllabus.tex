\documentclass[10pt]{article}

\usepackage{titlesec}
\titlespacing\section{0pt}{5pt}{2pt}
\titlespacing\subsection{0pt}{5pt}{2pt}
%\setlength{\parindent}{0pt} % remove automatic indentation


\usepackage[text={6.5in,8.5in},centering]{geometry}
\geometry{verbose,a4paper,tmargin=2.4cm,bmargin=2.4cm,lmargin=2.4cm,rmargin=2.4cm}

\usepackage{soul} % for \st{} strike-out
\usepackage{hyperref} % for activation hyperlinks
\hypersetup{colorlinks = true, allcolors = blue  }

\usepackage{enumitem}

\usepackage{booktabs}
\usepackage{array}
\usepackage{enumitem}
\usepackage{parskip}



%%%%%%%%%%%%%%%%%%%%%%%%%%%%%%%

\title{IB 514 SCIENTIFIC WRITING}

\author{}
\date{}

%%%%%%%%%%%%%%%%%%%%%%%%%%%%%%%

\begin{document}
\maketitle
\vspace{-50pt}


\section*{Meeting Times \& Location}
\noindent
MW 12pm-1:20pm in Cordley Hall 2406

\section*{Course Description}
\noindent
	Develops skills and strategies for producing clear, effective scientific manuscripts and 
	professional documents. Covers the full writing process, including writing habits, workflow 
	management, journal selection, and the peer review and publication process. Explores best 
	practices for integrating literature, using reference managers, and adopting alternative writing and 
	word-processing tools. Emphasizes practical, hands-on experience through real-world scientific 
	writing, peer feedback, and collaborative accountability.
		\textit{3 credits}

	
\section*{Prerequisites}
\noindent
	Officially, only graduate standing (or instructor approval).
	Unofficially, students must be ready to start writing a scientific manuscript the first week of the 
	term. What does “be ready” mean? That’s for you and your advisor(s) to decide. That said, being 
	ready does not mean being 100\% done with your analyses or having all your final figures drafted. 
	(A key lesson of the course is that writing can and should start much earlier than that.) However, 
	you do need to be far enough along to know what your paper’s main story arc and primary 
	message will be. (Feel free to contact me if you’re unsure.)

\section*{Learning Outcomes}
\noindent
After successful completion of this course, you should be able to:

\begin{itemize}[leftmargin=*]
	
	\item
	Design and implement a sustainable writing workflow that includes goal-setting, time 
	management, journal selection, and navigation of the peer review and publication process.
	\textit{Verified through: Writing plans, workflow reflections, journal targeting assignments, and 
	participation in structured writing/accountability activities.}
	
	\item
	Integrate and synthesize primary literature effectively by selecting relevant sources, accurately 
	representing prior work, and managing citations using a reference manager.
	\textit{Verified through: Annotated drafts, reference library checks, and citation accuracy in the 
	final manuscript.}
	
	\item
	Critically revise their own and others’ scientific writing by diagnosing problems in clarity, 
	argumentation, organization, and style, and by implementing substantive revisions in response to 
	peer and instructor feedback.
	\textit{Verified through: Peer-review assignments, revision memos, and comparison of early and 
	revised drafts.}
	
	\item 
	Produce a complete, submission-ready scientific manuscript that is clearly written, logically 
	structured, and aligned with the conventions and audience expectations of a target journal.
	\textit{Verified through: Drafts and final manuscript submission}.

\end{itemize}

\section*{Instructor \& Office Hours}
	\noindent
	Mark Novak\\
	Office: Cordley 5323\\
	Email: \href{mailto:mark.novak@oregonstate.edu}{mark.novak@oregonstate.edu}

\noindent
	I am happy to meet by appointment, or please feel free to stop by my office 
	anytime.

\section*{Course Materials \& Schedule}
\noindent
See
\href{https://github.com/ScientificWriting}{https://github.com/ScientificWriting}.
 

\section*{Course Work}
\noindent
There will be no exams, tests, or quizzes.
I will not grade your work because there won't be ``correct'' answers to the 
problems you will solve.
In fact, my goal is for you to not have almost no ``homework'' (besides readings 
and making progress on your manuscript writing).
However, the course \textit{will} require a significant allocation of time for 
thought and reflection outside of class.
There \textit{will} also be readings for in-class discussion to be read before class (see Schedule on GitHub repository).
Finally, you will need to complete all assigned in-class work, ideally by the end 
of the week in which the topic is covered but definitely before the end of Week 10.



\section*{Collaboration}
\noindent
My hope is for you to make a tremendous amount of progress on your 
manuscript during this course.
Your work will therefore, almost by definition and necessity, be your own.
That said, I \textit{strongly} encourage everyone to work as a team in
whatever ways are possible!
That's because what this course is all about is to provide you with tools.
Tools have to be learned and practiced.
Moreover, there's nothing better for learning to understand something than 
(successfully) explaining how you think about it to someone else.
Therefore, you are \textit{strongly} encouraged to collaborate.
This will be emphasized throughout the quarter by getting you to work or 
discuss your work in pairs or teams.

\section*{Student Learning Experience Survey}
\noindent
The online Student Learning Experience surveys will open to you the 
Wednesday of week 9 and close the Sunday before Finals Week. 
You will receive notification, instructions, and the link through your ONID email. 
You may also log into the survey via MyOregonState or directly at 
\url{https://beav.es/Student-Learning-Survey}. 

\begin{center}
\textit{
\textbf{Survey results are extremely important to how my department values 
this course.
They also help me improve the course and the learning experience of future 
students.}}
\end{center}

\noindent
Responses are anonymous (unless you choose to ``sign'' your comments, 
agreeing to relinquish anonymity of written comments) and are not available to 
instructors until after grades have been posted. 
I've been told that the results of scaled (quantitative) questions and signed 
comments go to me and the Head 
of my department.
Anonymous (unsigned) comments go only to me.

	
\section*{Evaluation of Student Performance}
\noindent
Grading (A-F) will be based on your participation, 
your project presentations 
and the completion of the various in-class tasks assigned throughout the quarter. 
Writing assignments will be graded on \textit{writing as a professional scientific practice}, not just 
prose quality.
Strong performance reflects clarity, revision, strategic decision-making, and engagement with the 
full publication process.
Accommodations for missed classes (e.g., due to fieldwork or conference attendance) are of course 
okay if we can identify a path to your having fully engaged in the course.
Please discuss your specific situation with me during Week 1.
Everyone who puts the effort into the class will get a grade that will make them happy (if you, as a 
graduate student, still care about grades).


\textit{Schema:} 
A = 91 \%+, 
B = 81--90 \%, 
C = 71--80 \%, 
D = 61--70 \%, 
F = 0--60 \%.
	
	Across all assignments, grading emphasizes four recurring dimensions:
	
	\begin{itemize}[leftmargin=*]
		\item \textbf{Clarity \& Structure}\\
		Writing is logically organized, well signposted, and appropriate for the target audience.
		Manuscripts follow conventions of the intended format (e.g., IMRaD, review, data paper).
		
		\item \textbf{Revision \& Process}\\
		Evidence of \emph{substantive revision}, not just copy-editing.
		Feedback is incorporated thoughtfully and revisions are justified.
		
		\item \textbf{Scholarly Practice}\\
		Literature is synthesized (not listed), citations are accurate and managed with a reference 
		manager,
		and figures communicate ideas clearly and professionally.
		
		\item \textbf{Professionalism}\\
		Work aligns with journal expectations, peer reviews are constructive and specific,
		and collaboration, authorship, and AI use are handled transparently.
	\end{itemize}
	
	\vspace{0.5em}
	
	\section*{Performance Levels}
	
	\begin{tabular}{p{0.22\textwidth} p{0.73\textwidth}}
		\toprule
		\textbf{Level} & \textbf{Description} \\
		\midrule
		Exemplary &
		Publication-oriented work; clear, strategic, and polished; strong evidence of revision and
		intentional decision-making. \\
		Satisfactory &
		Solid graduate-level work with minor weaknesses or inconsistencies. \\
		Needs Improvement &
		Incomplete, unclear, poorly revised, or misaligned with assignment expectations. \\
		\bottomrule
	\end{tabular}
	
	\vspace{0.75em}
	
	\section*{Assignment Weights \& Focus}
	
	\begin{tabular}{p{0.5\textwidth} p{0.15\textwidth} p{0.3\textwidth}}
		\toprule
		\textbf{Assignment} & \textbf{Weight} & \textbf{Primary Focus} \\
		\midrule
		Writing Workflow \& Habits Portfolio & 15\% & Sustainable writing practices \\
		Project Proposal \& Journal Targeting & 15\% & Scope, audience, planning \\
		Annotated Manuscript Drafts & 30\% & Drafting, structure, revision \\
		Peer Review \& Collaboration Portfolio & 15\% & Feedback and ethics \\
		Final Manuscript \& Reflection & 25\% & Submission-ready writing \\
		\bottomrule
	\end{tabular}
	
	\vspace{0.75em}
	
	\subsection*{What ``Submission-Ready'' Means}
	
	A submission-ready manuscript is formatted for a specific journal, has a clear narrative and 
	audience awareness, uses figures effectively, meets word limits strategically, and could  
	realistically be submitted after this course.
	Perfection is \textbf{not} required, but professionalism is.
	


\noindent\rule[0.5ex]{\linewidth}{1pt}


\section*{Policies}
\small

\subsection*{Academic Calendar}
\noindent
All students are subject to the registration and refund deadlines as stated in the Academic Calendar: 
\url{https://registrar.oregonstate.edu/osu-academic-calendar}

\subsection*{Equity, Justice and Inclusion}
\noindent
Oregon State University, and the Department of Integrative Biology, have a lot 
of work to do on improving the level of equity, justice and inclusion in our 
community. 
I will take responsibility for creating an welcoming and supportive climate for 
everyone in the course. 
If you have concerns or suggestions related to these issues as they pertain to 
the course, please do not hesitate to contact me and/or the Office for 
Institutional Diversity
(\href{https://diversity.oregonstate.edu/}{https://diversity.oregonstate.edu/}).

\subsection*{Student Conduct Expectations}
\noindent
See \href{https://beav.es/codeofconduct}{https://beav.es/codeofconduct}.


\subsection*{Student Bill of Rights}
\noindent
OSU has twelve established student rights. They include due process in all university disciplinary 
processes, an equal opportunity to learn, and grading in accordance with the course syllabus:
\href{https://asosu.oregonstate.edu/advocacy/rights}{https://asosu.oregonstate.edu/advocacy/rights}.


\subsection*{Students with Disabilities}
\noindent
Accommodations for students with disabilities are determined and approved by 
Disability Access Services (DAS).
If you, as a student, believe you are eligible for accommodations but have not 
obtained official approval, please contact DAS at 541-737-4098 or at
\href{http://ds.oregonstate.edu}{http://ds.oregonstate.edu}.
DAS notifies students and faculty members of approved academic 
accommodations and coordinates implementation of those accommodations.
While not required, students and faculty members are encouraged to discuss 
details of the implementation of individual accommodations.

\subsection*{Reach Out for Success}
\noindent
University students encounter setbacks from time to time. If you encounter difficulties and need 
assistance, it’s important to reach out. Consider discussing the situation with an instructor or 
academic advisor. Learn about resources that assist with wellness and academic success at 
\href{http://oregonstate.edu/ReachOut}{http://oregonstate.edu/ReachOut}. 
If you are in immediate crisis, please call or text the Suicide \& Crisis Lifeline at 988.
For financial hardship:
Any student whose academic performance is impacted by financial stress or the inability to afford 
groceries, housing or other necessities, for any reason, is urged to contact the Office of Student 
Care (541-737-8748).
To find even more resources, check out the Student Resources Guide for additional support services 
and guidance.

\subsection*{Religious Holidays}
\noindent
Oregon State University strives to respect all religious practices.
If you have religious holidays that are in conflict with any of the requirements of 
this class, please see me as soon as you can so that we can make alternative 
arrangements.


\end{document}
